\documentclass[a4paper,12pt,twoside]{scrreprt}
% Autor der Vorlage: Klaus Rheinberger, FH Vorarlberg
% 2017-02-20

%% Hilfe: z.B.
% empfohlener Einstieg: http://latex.tugraz.at/  
% https://de.wikibooks.org/wiki/LaTeX-Kompendium:_Schnellkurs:_Erste_Schritte
% https://de.wikibooks.org/wiki/LaTeX-Kompendium:_Schnellkurs
% https://de.wikibooks.org/wiki/LaTeX-Kompendium

%% Pakete:
% Der Befehl \usepackage[latin9]{inputenc} ermöglicht die direkte Angabe von Umlauten. Übrigens lässt sich so auch das Euro-Zeichen direkt eingeben. Auf Betriebssystemen, wie zum Beispiel allen neueren Linux-Distributionen, verwendet man statt \usepackage[latin9]{inputenc} besser \usepackage[utf8]{inputenc}, auf Applesystemen verwendet man \usepackage[macce]{inputenc} (oder das für ältere Modelle gültige \usepackage[applemac]{inputenc}).
\usepackage[utf8]{inputenc}
\usepackage[T1]{fontenc}    % Silbentrennung bei Sonderzeichen
\usepackage{graphicx}       % Bilder einbinden
\usepackage[ngerman]{babel} % Deutsche Sprachanpassungen
\usepackage{csquotes}       % When using babel or polyglossia with biblatex, loading csquotes is recommended to ensure that quoted texts are typeset according to the rules of your main language.
\usepackage{acronym}  % für optionales Abkürzungsverzeichnis
\usepackage{eurosym}  % z. B. \EUR{12345,68}
\usepackage[linktocpage=true]{hyperref} % Links z. B. \href{https://www.wikibooks.org}{Wikibooks home}
\usepackage[bindingoffset=8mm]{geometry}% Bindeverlust von 8mm einbeziehen. Mit dem geometry-Paket können Sie die Ränder auch ganz individuell anpassen.
\usepackage{caption} % Abbildungslegenden
\captionsetup{format=hang, justification=raggedright}

\usepackage[style=alphabetic,citestyle=alphabetic,backend=biber]{biblatex}   % Literaturverweise
%\usepackage[style=numeric,citestyle=numeric,backend=biber]{biblatex}
% biblatex comes with a variety of built-in bibliography/citation style families (numeric, alphabetic, authoryear, authortitle, verbose), and there's a growing number of custom styles:
% https://de.sharelatex.com/learn/Biblatex_citation_styles
% https://de.sharelatex.com/learn/Biblatex_bibliography_styles
\addbibresource{Zotero-Beispiele.bib}    % Zotero-Beispiele.bib ist die verwendete Bibtex-Datei 
% Anstatt die Bibtex-Datei selber zu erstellen, kann sie z. B. aus einer Zotero-Sammlung zu BibTeX exportiert werden.


%% Einstellungen:
\setcounter{secnumdepth}{4}
\setcounter{tocdepth}{4}   % Tiefe der Gliederung im In haltsverzeichnis


%% ERSETZEN VON ECKIGEN KLAMMERN:
% Ersetzen Sie den Text in den eckigen Klammern!

\begin{document}


% Titelblatt:
% \newpage\mbox{}\newpage
\cleardoublepage   % force output to a right page
\thispagestyle{empty}
\begin{titlepage}
  \begin{flushright}
  \includegraphics[width=0.4\linewidth]{Logo-A3}
  \end{flushright}
  \begin{flushleft}
  \section*{Lerntagebuch}
  \subsection*{Software Lebenszyklus und Qualität}

  \vspace{12.5cm}
  

  Fachhochschule Vorarlberg\newline

  \vspace{0.5cm}
  
  \vspace{0.5cm}
  
  Vorgelegt von\newline
  Christina Tschol\newline
  Florian Bechtold\newline
  Dornbirn, Jänner 2019
  \end{flushleft}
\end{titlepage}


% Inhaltsverzeichnis:
\cleardoublepage   % force output to a right page
\tableofcontents

\clearpage
\phantomsection
\addcontentsline{toc}{chapter}{Abbildungsverzeichnis}
\listoffigures

%% Die Kapitelstruktur ist mit der betreuungsperson abzustimmen!

\chapter{Einleitung}
Das Lerntagebuch ist gegliedert nach den vier stattgefundenen Vorlesungen im November bzw. Dezember. Die in den jeweiligen Vorlesungen präsentierten Folien und Inhalte wurden von uns nach der
Vorlesung noch einmal wiederholt und in das Tagebuch abgelegt. Dabei soll erwähnt werden, dass nicht alle Inhalte der Vorlesung in dem Tagebuch beinhaltet sind und somit keine Zusammenfassung
der Vorlesung Software Lifecycle und Qualität sein soll. So wurden einzelne, für uns teilweise in der Vorlesung unklare Themen, teils über weitere Recherchen weiter ausgeführt. 
\newline
Die Hausübung zum Thema der \textbf{Quadratischen Gleichung} wird am Ende des Lerntagebuches vorgestellt und erläutert. Erklärungen zu verschiedensten Abkürzungen und Begrifflichkeiten sind im Glossar am Ende des Lerntagebuches zu finden.


\chapter{Vorlesung 1 (28.11.2018)}
In der ersten Vorlesung am 28.11.2018 wurden Teile der erste Foliensatz \textbf{SWE P01 Introduction} und Teile des zweiten Foliensatzes \textbf{SWE P02 Motivation} präsentiert. 
  \section{Foliensatz SWE P01 Introduction}
    \subsection{Opening Remarks}
    \subsection{Vorstellung von Literatur}
    \subsection{Kurzer Überblick über Kursinhalte}

  \section{Foliensatz SWE P02 Motivation}
    \subsection{Observations}
    \subsection{Selected Events}
    \subsection{Important Terms and Concepts}
    \subsection{Kurze Übung in der Lehrveranstaltung}
    \subsection{Beginn mit Sociotechnical Systems}
      \subsubsection{Terminologie}
      \subsubsection{Systemattribute}
      \subsubsection{Lifecycle V Methode}
      \subsubsection{Environmental Influences}

\chapter{Vorlesung 2 (5.12.2018)}

\chapter{Vorlesung 3 (12.12.2018)}

\chapter{Vorlesung 4 (19.12.2018)}

\chapter{Hausübung}

% evtl. Abkürzungsverzeichnis:
\clearpage
\phantomsection
\addcontentsline{toc}{chapter}{Glossar}  % evtl. ersetzen durch \addcontentsline{toc}{chapter}{Abkürzungsverzeichnis}
\chapter*{Glossar} % evtl. ersetzen durch \chapter*{Abkürzungsverzeichnis}
\begin{acronym}
 \acro{ETW}{Energietechnik und Energiewirtschaft}
 \acro{SQL}{Structured Query Language}
 \acro{Bash}{Bourne-again shell}
\end{acronym}


\end{document}
